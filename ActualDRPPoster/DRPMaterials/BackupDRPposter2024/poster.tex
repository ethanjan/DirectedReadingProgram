% Unofficial University of Oxford Poster Template
% https://github.com/andiac/gemini-cam
% a fork of https://github.com/anishathalye/gemini

\documentclass[final]{beamer}

% ====================
% Packages
% ====================

\usepackage[T1]{fontenc}
\usepackage{lmodern}
\usepackage[size=custom,width=120,height=72,scale=1.0]{beamerposter}
\usetheme{gemini}
\usecolortheme{ox}
\usepackage{graphicx}
\usepackage{booktabs}
\usepackage{tikz}
\usepackage{pgfplots}
\pgfplotsset{compat=1.14}
\usepackage{anyfontsize}

% ====================
% Lengths
% ====================

% If you have N columns, choose \sepwidth and \colwidth such that
% (N+1)*\sepwidth + N*\colwidth = \paperwidth
\newlength{\sepwidth}
\newlength{\colwidth}
\setlength{\sepwidth}{0.025\paperwidth}
\setlength{\colwidth}{0.3\paperwidth}

\newcommand{\separatorcolumn}{\begin{column}{\sepwidth}\end{column}}

% ====================
% Title
% ====================
\title{An Overdetermined Problem in Symmetry}

\author{Ethan Martirosyan \and Yingpeng He \and Mentor: Jihye}

\institute[shortinst]{University of California Santa Barbara}

% ====================
% Footer (optional)
% ====================

%\footercontent{
 % \href{https://www.example.com}{https://www.example.com} \hfill
 % ABC Conference 2025, London --- XYZ-1234 \hfill
  %\href{mailto:john.doe@example.com}{john.doe@example.com}}
% (can be left out to remove footer)

% ====================
% Logo (optional)
% ====================
% Refer to https://github.com/k4rtik/uchicago-poster
% logo: https://communications.admin.ox.ac.uk/communications-resources/visual-identity/identity-guidelines/the-oxford-logo
% use this to include logos on the left and/or right side of the header:
\logoright{\includegraphics[height=7cm]{logos/ox-logo.png}}
% \logoleft{\includegraphics[height=7cm]{logos/ox-logo.png}}

% ====================
% Body
% ====================

\begin{document}



\begin{frame}[t]
\begin{columns}[t]
\separatorcolumn

\begin{column}{\colwidth}

  \begin{block}{Background}

    Let us consider the following problem. Let $\Omega \subseteq \mathbb{R}^n$ be domain that is bounded, open, and connected. Furthermore, suppose that the boundary $\partial{\Omega}$ of $\Omega$ is smooth. Let $u: \Omega \rightarrow \mathbb{R}$ be a function that satisfies the following conditions: $\Delta u = -1$ in $\Omega$, $u=0$ on $\partial{\Omega}$, and $\frac{\partial{u}}{\partial{n}} = c$ on $\Omega$ for some constant $c$. Then, $\Omega$ must be a ball. Furthermore, we know that $u(x) = (b^2-r^2)/2n$, where $b$ is the ball's radius and $r$ is the distance to its center.

  This theorem has many applications in physics. For example, we may consider an incompressible fluid moving through a straight pipe of cross sectional form $\Omega$. If we fix a rectangular coordinate system with the $z$ axis in the same direction as the pipe, then the velocity $u$ depends only on $x$ and $y$, and it satisfies the differential equation $\Delta u = -A$ for some constant $A$. Furthermore, because the fluid is viscous, we know that $u = 0$ on $\partial{\Omega}$; that is, there is no movement on the boundary of the pipe. Finally, we note that $\mu\partial{\mu}/\partial{n}$ is the tangential stress on the pipe wall, where $\mu$ is viscosity constant. If the tangential stress is constant, then we may apply the above theorem to conclude that $\Omega$ is a circular cross section.
 
 This theorem can also be applied 

  \end{block}

  \begin{block}{First Proof}

    We will first prove this theorem by the moving plane method. Let $T_0$ be a $n-1$ dimensional hyperplane in $\mathbb{R}^n$ that does not intersect the domain $\Omega$. We begin to move this plane until it intersects $\Omega$. When this occurs, the new plane $T$ splits $\Omega$ into two pieces. The piece of $\Omega$ that lies on the same side of $T$ as our initial plane $T_0$ is denoted by $\Sigma(T)$. We reflect $\Sigma(T)$ in $T$ to obtain $\Sigma^\prime(T)$. As $T$ is moved through $\Omega$, it is evident that $\Sigma^\prime(T)$ will remain in $\Omega$ unless one of the following two events occurs:
    \begin{enumerate}[]
        \item The set $\Sigma^\prime(T)$ meets $\Omega$ at a point $P$
        \item $T$ becomes orthogonal to $\Omega$ at some point $Q$
    \end{enumerate} When this occurs, we stop moving the plane $T$, and we denote the resulting plane by $T^\prime$.
    
    We claim that $\Omega$ is symmetric about $T^\prime$. This is crucial because proving this would also prove the theorem by extension. To see how, we recall that the plane $T$ was chosen arbitrarily. If we can show that $\Omega$ is symmetric about $T^\prime$, then we have shown that $\Omega$ is symmetric in all possible directions. Since $\Omega$ is simply connected and maintains this strong symmetry property, it must be a ball.
    
    For convenience, let us denote $\Sigma^\prime := \Sigma^\prime(T)$. In order to show that this symmetry property holds, we introduce a new function $v: \Sigma^\prime \rightarrow \mathbb{R}$ defined as follows: $v(x) = u(x^\prime)$ for $x \in \Sigma^\prime$, where $x^\prime$ is obtained by reflecting $x$ across $T^\prime$. If we can show that $u = v$ in $\Sigma^\prime$, it will follow that $\Omega$ is symmetric about $T^\prime$. First, we note some properties of $v$ that can easily be obtained from the corresponding properties of $u$. It can easily be seen that $\Delta v = -1$ in $\Sigma^\prime$, that $v = u$ on the plane $T^\prime$, that $v = 0$ and $\partial{v}/\partial{n} = c$ on the boundary of $\Sigma^\prime$. Using these facts, we deduce that $\Delta(u-v) = 0$ in $\Sigma^\prime$ and that $u-v \geq 0$ on the boundary of $\Sigma^\prime$. By the Maximum Principle, we have $u - v > 0$ at every point in $\Sigma^\prime$ or $u-v = 0$ in $\Sigma^\prime$. As stated above, we are trying to prove that the latter is true. Thus, we must prove that $u-v > 0$ cannot occur. For the sake of contradiction, let us suppose that $u - v > 0$ in $\Sigma^\prime$. First, we suppose that $\Sigma^\prime$ is internally tangent to $\Omega$ at some point $P$. By the definitions of $u$ and $v$, we have $u - v = 0$ at $P$. Appealing to the boundary point maximum principle, we find that $\frac{\partial}{\partial{n}}(u-v) > 0$ at $P$. However, we previously established that $\partial{u}/\partial{n} = \partial{v}/\partial{n} = c$. Thus, we have reached a contradiction. Next, we consider the case in which $T^\prime$ is orthogonal to the boundary of $\Omega$ at some point $Q$. In this case, we cannot apply the boundary point maximum principle directly because there is no ball internally tangent to $\Sigma^\prime$ at $Q$. 

\end{block}

  
\end{column}

\separatorcolumn

\begin{column}{\colwidth}

  \begin{block}{A block containing an enumerated list}

    Vivamus congue volutpat elit non semper. Praesent molestie nec erat ac
    interdum. In quis suscipit erat. \textbf{Phasellus mauris felis, molestie
    ac pharetra quis}, tempus nec ante. Donec finibus ante vel purus mollis
    fermentum. Sed felis mi, pharetra eget nibh a, feugiat eleifend dolor. Nam
    mollis condimentum purus quis sodales. Nullam eu felis eu nulla eleifend
    bibendum nec eu lorem. Vivamus felis velit, volutpat ut facilisis ac,
    commodo in metus.

    \begin{enumerate}
      \item \textbf{Morbi mauris purus}, egestas at vehicula et, convallis
        accumsan orci. Orci varius natoque penatibus et magnis dis parturient
        montes, nascetur ridiculus mus.
      \item \textbf{Cras vehicula blandit urna ut maximus}. Aliquam blandit nec
        massa ac sollicitudin. Curabitur cursus, metus nec imperdiet bibendum,
        velit lectus faucibus dolor, quis gravida metus mauris gravida turpis.
      \item \textbf{Vestibulum et massa diam}. Phasellus fermentum augue non
        nulla accumsan, non rhoncus lectus condimentum.
    \end{enumerate}

  \end{block}

  \begin{block}{Fusce aliquam magna velit}

    Et rutrum ex euismod vel. Pellentesque ultricies, velit in fermentum
    vestibulum, lectus nisi pretium nibh, sit amet aliquam lectus augue vel
    velit. Suspendisse rhoncus massa porttitor augue feugiat molestie. Sed
    molestie ut orci nec malesuada. Sed ultricies feugiat est fringilla
    posuere.

    \begin{figure}
      \centering
      \begin{tikzpicture}
        \begin{axis}[
            scale only axis,
            no markers,
            domain=0:2*pi,
            samples=100,
            axis lines=center,
            axis line style={-},
            ticks=none]
          \addplot[red] {sin(deg(x))};
          \addplot[blue] {cos(deg(x))};
        \end{axis}
      \end{tikzpicture}
      \caption{Another figure caption.}
    \end{figure}

  \end{block}

  \begin{block}{Nam cursus consequat egestas}

    Nulla eget sem quam. Ut aliquam volutpat nisi vestibulum convallis. Nunc a
    lectus et eros facilisis hendrerit eu non urna. Interdum et malesuada fames
    ac ante \textit{ipsum primis} in faucibus. Etiam sit amet velit eget sem
    euismod tristique. Praesent enim erat, porta vel mattis sed, pharetra sed
    ipsum. Morbi commodo condimentum massa, \textit{tempus venenatis} massa
    hendrerit quis. Maecenas sed porta est. Praesent mollis interdum lectus,
    sit amet sollicitudin risus tincidunt non.

    Etiam sit amet tempus lorem, aliquet condimentum velit. Donec et nibh
    consequat, sagittis ex eget, dictum orci. Etiam quis semper ante. Ut eu
    mauris purus. Proin nec consectetur ligula. Mauris pretium molestie
    ullamcorper. Integer nisi neque, aliquet et odio non, sagittis porta justo.

    \begin{itemize}
      \item \textbf{Sed consequat} id ante vel efficitur. Praesent congue massa
        sed est scelerisque, elementum mollis augue iaculis.
        \begin{itemize}
          \item In sed est finibus, vulputate
            nunc gravida, pulvinar lorem. In maximus nunc dolor, sed auctor eros
            porttitor quis.
          \item Fusce ornare dignissim nisi. Nam sit amet risus vel lacus
            tempor tincidunt eu a arcu.
          \item Donec rhoncus vestibulum erat, quis aliquam leo
            gravida egestas.
        \end{itemize}
      \item \textbf{Sed luctus, elit sit amet} dictum maximus, diam dolor
        faucibus purus, sed lobortis justo erat id turpis.
      \item \textbf{Pellentesque facilisis dolor in leo} bibendum congue.
        Maecenas congue finibus justo, vitae eleifend urna facilisis at.
    \end{itemize}

  \end{block}

\end{column}

\separatorcolumn

\begin{column}{\colwidth}

  \begin{exampleblock}{A highlighted block containing some math}

    A different kind of highlighted block.

    $$
    \int_{-\infty}^{\infty} e^{-x^2}\,dx = \sqrt{\pi}
    $$

    Interdum et malesuada fames $\{1, 4, 9, \ldots\}$ ac ante ipsum primis in
    faucibus. Cras eleifend dolor eu nulla suscipit suscipit. Sed lobortis non
    felis id vulputate.

    \heading{A heading inside a block}

    Praesent consectetur mi $x^2 + y^2$ metus, nec vestibulum justo viverra
    nec. Proin eget nulla pretium, egestas magna aliquam, mollis neque. Vivamus
    dictum $\mathbf{u}^\intercal\mathbf{v}$ sagittis odio, vel porta erat
    congue sed. Maecenas ut dolor quis arcu auctor porttitor.

    \heading{Another heading inside a block}

    Sed augue erat, scelerisque a purus ultricies, placerat porttitor neque.
    Donec $P(y \mid x)$ fermentum consectetur $\nabla_x P(y \mid x)$ sapien
    sagittis egestas. Duis eget leo euismod nunc viverra imperdiet nec id
    justo.

  \end{exampleblock}

  \begin{block}{Nullam vel erat at velit convallis laoreet}

    Class aptent taciti sociosqu ad litora torquent per conubia nostra, per
    inceptos himenaeos. Phasellus libero enim, gravida sed erat sit amet,
    scelerisque congue diam. Fusce dapibus dui ut augue pulvinar iaculis.

    \begin{table}
      \centering
      \begin{tabular}{l r r c}
        \toprule
        \textbf{First column} & \textbf{Second column} & \textbf{Third column} & \textbf{Fourth} \\
        \midrule
        Foo & 13.37 & 384,394 & $\alpha$ \\
        Bar & 2.17 & 1,392 & $\beta$ \\
        Baz & 3.14 & 83,742 & $\delta$ \\
        Qux & 7.59 & 974 & $\gamma$ \\
        \bottomrule
      \end{tabular}
      \caption{A table caption.}
    \end{table}

    Donec quis posuere ligula. Nunc feugiat elit a mi malesuada consequat. Sed
    imperdiet augue ac nibh aliquet tristique. Aenean eu tortor vulputate,
    eleifend lorem in, dictum urna. Proin auctor ante in augue tincidunt
    tempor. Proin pellentesque vulputate odio, ac gravida nulla posuere
    efficitur. Aenean at velit vel dolor blandit molestie. Mauris laoreet
    commodo quam, non luctus nibh ullamcorper in. Class aptent taciti sociosqu
    ad litora torquent per conubia nostra, per inceptos himenaeos.

    Nulla varius finibus volutpat. Mauris molestie lorem tincidunt, iaculis
    libero at, gravida ante. Phasellus at felis eu neque suscipit suscipit.
    Integer ullamcorper, dui nec pretium ornare, urna dolor consequat libero,
    in feugiat elit lorem euismod lacus. Pellentesque sit amet dolor mollis,
    auctor urna non, tempus sem.

  \end{block}

  \begin{block}{References}

    \nocite{*}
    \footnotesize{\bibliographystyle{plain}\bibliography{poster}}

  \end{block}

\end{column}

\separatorcolumn
\end{columns}
\end{frame}

\end{document}
